%%%%%%%%%%%%%%%%%%%%%%%%%%%%%%%%%%%%%%%%%
% Academic Title Page
% LaTeX Template
% Version 2.0 (17/7/17)
%
% This template was downloaded from:
% http://www.LaTeXTemplates.com
%
% Original author:
% WikiBooks (LaTeX - Title Creation) with modifications by:
% Vel (vel@latextemplates.com)
%
% License:
% CC BY-NC-SA 3.0 (http://creativecommons.org/licenses/by-nc-sa/3.0/)
% 
%%%%%%%%%%%%%%%%%%%%%%%%%%%%%%%%%%%%%%%%%


	\newcommand{\HRule}{\rule{\linewidth}{0.5mm}} % Defines a new command for horizontal lines, change thickness here
	
	\center % Centre everything on the page
	
	%------------------------------------------------
	%	Headings
	%------------------------------------------------
	
	\textsc{\LARGE Center of Excellence SDC}\\[1.5cm] % Main heading such as the name of your university/college
	
	%\textsc{\Large Major Heading}\\[0.5cm] % Major heading such as course name
	
	%\textsc{\large Minor Heading}\\[0.5cm] % Minor heading such as course title
	
	%------------------------------------------------
	%	Title
	%------------------------------------------------
	
	\HRule\\[0.4cm]
	
	{\huge\bfseries\setstretch{1.5} \rule[-0.25cm]{0pt}{0.25cm}$title$}\\[0.4cm] % Title of your document
	{\large\bfseries\setstretch{1.5} $subtitle$}\\[0.4cm]
	{\large\monthyeardate\today}\\[0.4cm] % Date, change the \today to a set date if you want to be precise

	\HRule\\[1.5cm]
	
	%------------------------------------------------
	%	Author(s)
	%------------------------------------------------
	
	\begin{minipage}{0.4\textwidth}
		\begin{flushleft}
			\large
			\textit{Authors}\\
			Anco \textsc{Hundepool}\\
    		Josep \textsc{Domingo-Ferrer}\\
    		Luisa \textsc{Franconi}\\
    		Sarah \textsc{Giessing}\\
    		Rainer \textsc{Lenz}\\
    		Jane \textsc{Naylor}\\
    		Eric \textsc{Schulte Nordholt}\\
    		Giovanni \textsc{Seri}\\
    		Peter-Paul \textsc{De Wolf}\\
    		Reinhard \textsc{Tent}\\
    		Andrzej \textsc{Młodak}\\
    		Johannes \textsc{Gussenbauer}\\
    		Kamil \textsc{Wilak}
		\end{flushleft}
	\end{minipage}
	~
	\begin{minipage}{0.4\textwidth}
		\begin{flushright}
			\large
			%\textit{Supervisor}\\
			%Dr. Caroline \textsc{Becker} % Supervisor's name
		\end{flushright}
	\end{minipage}
	
	% If you don't want a supervisor, uncomment the two lines below and comment the code above
	%{\large\textit{Author}}\\
	%John \textsc{Smith} % Your name
	
	%------------------------------------------------
	%	Date
	%------------------------------------------------
	
	\vfill\vfill\vfill % Position the date 3/4 down the remaining page
	
	%------------------------------------------------
	%	Logo
	%------------------------------------------------
	
	\vfill\vfill
	\includegraphics[width=0.5\textwidth]{Images/media/eu_funded.png}\\[1cm] % Include a department/university logo - this will require the graphicx package
	 
	%----------------------------------------------------------------------------------------
	
	\vfill % Push the date up 1/4 of the remaining page
	
	\newpage 

	

		
	\noindent \textbf{\large Background}\bigskip

	\justifying

	\noindent The first edition of this handbook was written in the framework of the Centre of Excellence on SDC launched in 2005 (grant agreement No 25200.2005.001-2005.619).
 	 This second edition was developed in th eframework of the project on Statistical Methods and Tools
	 for Time Series, Seasonal Adjustment and Statistical Disclosure Control (in short: STACE), 
	 co-financed by the European Union by means of grant agreement 899218 -- 2019-BG-Methodology launched in 2020. 
	 Work package 2 of this project concerned a Center of Excellence on Statistical Disclosure Control (CoE on SDC)
	  which was formed by the national statistical institutes (NISs) of The Netherlands, Germany, France, Austria, 
	  Iceland, Slovenia, Poland and Bulgaria.

	One of the goals of this CoE was to provide three guidelines for applying SDC. 
	The CoE, together with Eurostat and the European Expert Group on Statistical Disclosure Control, 
	decided on three topics for these guidelines: (1)~\textit{Guidelines for SDC methods for Census and Demographics Data},
	 (2)~\textit{Guidelines for SDC Methods Applied on Geo-Referenced Data} and (3)~\textit{Update of the General SDC Handbook}.

	A subgroup of the CoE on SDC was formed by the NSIs of Germany, Austria, France, The Netherlands, Poland and Slovenia. 
	This subgroup worked on the \textit{Update of the General SDC Handbook}. The current document is the result of this work.

	\vfill
